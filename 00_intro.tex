\chapter{はじめに}

本書は高校〜大学初級の数学教材です。中学数学を完全に
習得している読者を想定し, 筑波大学生物資源学類の
1年次の基礎数学・物理学・数理科学演習・統計学入門等の
授業で使います。

高校数III未習の人は多分, 中学数学も抜けてるでしょうから, 
まず中学数学を復習してから取り組もう。そして, デキる友人や
教員にたくさん質問し, 人よりも努力しよう。

高校数III既習の人は, ナメてかからないで, 謙虚に
勉強し直そう。一見簡単そうに見えても手強いよ。

\section*{正しい勉強の鉄則}

正しい方法でやらねば勉強は身につきません。以下の鉄則を
守ろう。\\

\subsection*{鉄則1: 毎日やる!}
忙しくても, 疲れていても, 旅行中でも, どんなときでも
少しでいいから勉強を続ける。途切れさせない。休んでいいのは, 
病気や怪我でドクターストップがかかったときだけ。\\

\subsection*{鉄則2: 丁寧に!}
その場しのぎの雑な勉強(つまみ食い・読み飛ばし)は何も
身につかない。急がばまわれ。全ての解説を読み, 実際に
鉛筆を持って, 全ての証明と例を再現し, 問題を解く。
わかったこととわからないことを明確に区別し, 確実に
わかるところに戻る。証明や計算は, 途中を飛ばさない。\\

\subsection*{鉄則3: 理解する!}
理解しないと何も残りません。「わかんない」「めんどくさい」から
「解き方」や「答」だけをやみくもに覚える, という勉強をする人が
いますが, かえって効率悪いし, 数学が嫌いになるだけです。\\

\subsection*{鉄則4: 再現する!}
理解したら, そこで満足しないで, 何も見ないでそれを紙の
上や頭の中で再現しよう。そうすると, 細かいところでまだ
わかっていなかった, ということが見つかるでしょう。それらを
潰していくのです。完璧に再現できるまでやりましょう。\\

\subsection*{鉄則5: 質問する!}
どうしてもわからなければ, 教員でも友人でも, とにかく
わかっている人に教えてもらおう。そういう「コミュ力」も立派な学力!\\

\subsection*{鉄則6: 定義を大切に!}
定義は, スポーツのルールや, 物語の登場人物
みたいなもの。覚えないと話になりません。「意味」とか
「イメージ」はその後。知らない人でも, 名前と顔を
覚えれば仲良くなれるのと同じ。そして, \textgt{困ったら定義に戻る}!

ちなみに, 定義は, 一字一句丸暗記するような
ものではなく, 論理的に同じなら, 覚えやすいように
適当に言い換えてもOK。定義の確認には巻末の索引を
活用しよう!\\

\subsection*{鉄則7: 紙の上で考える!}
数学が苦手な人は, 頭の中だけで考えて安易に暗算に頼ります。
数学は紙で考えるものです。頭の中のイメージや論理を紙の上に
可視化する。式変形や計算は暗算で済まさず, 途中経過も書く。
他の人が見てもわかるように, 整理して筋の通った解答を書く。
そうすればわかることが多いのです。そのために, 紙は贅沢に
使おう。\\

\subsection*{鉄則8: 誤植訂正は速攻で!}
大学教材には誤植はつきもの。この教材も入念にチェックしていますが, 
毎年必ず10個近くの誤植が見つかります。誤植訂正が出たら, すぐにテキスト
の該当部分を修正。\\

\subsection*{鉄則9: いらんことを考えない!}
この鉄則は, 以上の鉄則のまとめです。毎日やるのは, 
「今日は勉強しようかどうしようか」と悩む無駄を省くため。
丁寧にやれば, ケアレスミスが減り, 無駄に考えることが減る。
質問すれば, 簡単なことの見落としやダメな思い込みがわかる。
定義に戻って素直に考えればすんなりわかる。紙を使えば, 
脳の負担が軽減され, 本質的な部分に思考を集中できる。

ちなみに, 受験秀才は, なまじセンス(ひらめきや直感)が良いので, 
それに頼ることに慣れ過ぎてしまい, 大学では「いらんこと」
を考えて迷走する傾向があります。

\section*{本書の使い方}

\textgt{関数電卓を用意せよ}: 
実際に数値を計算する問題もあるので, 関数電卓(三角関数や指数・対数等が計算できる電卓)
を用意しましょう。高機能な電卓はとっつきにくいので, 安価でシンプルなもの
(といっても四則演算しかできないのはダメ)をお薦めします。いろんな授業で使います。
スマホやタブレットでも代用可能ですが, それらはテストでは使えません。\\

\textgt{参考書}: 基本的に不要。どうしてもというなら...

小中学校の復習をしたい人へ:「日本一わかりやすい数学の授業」
「日本一わかりやすい数学の授業2」「日本一わかりやすい数学の授業3」
(創拓社出版) ... 小学校高学年から中学までのレベルで, 雰囲気はこどもっぽい
ですが, 数学の本質を鋭く捉えている本です。

理科(物理など)の話題についていけない人へ:「発展コラム式 中学理科の
教科書 第1分野」(ブルーバックス) ... 中学理科をナメてはダメ。
この本を全部きちんと理解すれば, 本書を読むのに困らないでしょう。\\

%統計学をもっと丁寧に勉強したい人へ:「Rによるやさしい統計学」(オーム社) ... 
%統計学は, 理論を学ぶだけでなく, 実際のデータを計算機でいじってみる
%ことで, 身に付いてきます。この参考書は, 2年次の「統計学基礎演習」
%の教科書にもなっています。Rというのは有名な統計解析ソフト(無料)です。\\

\textgt{本書の構成}: 
\begin{itemize}
\item 高校数学を取捨選択し, 並べ替えました。特に, 早い段階
で微積分を学びます。授業ですぐに必要になるからです。一方で, 2次関数の
解の分離や平面図形などはばっさり落としました。

\item 高校と違う記号(ベクトルの太字表記等)を使ったり, 高校で学ばない
数学(対数グラフ, 不偏分散, 微分方程式, 集合の直積など)を
少し扱っています。

\item 数学類ではないので, 厳密性にはそれほど
こだわっていません。例えば微分積分の極限操作は, 
数学類で学ぶ理論($\epsilon-\delta$論法)には依らず, 
近似的・直感的な理解で良しとします。

\item そのかわり計算機(電卓やパソコン)をガンガン使います。
計算機で数値やグラフを実際にいじって納得することを大切にします。
そういう例や問は\textgt{必ず実際に自分で}計算機を操作してみて下さい。

\item 問題には解答をつけましたが, 一部の問題の解答は「略」としました
(解答が載っていない問題も同様に, 「解答略」と解釈して下さい。)。
理解せずにやみくもに解答を丸写しする先輩がこれまでにいたからです。
恨むなら先輩を恨んで下さい (\^{}\_\^{};)。「略」とした問題のほとんどは, 本文を
丁寧に読めば簡単にわかるものです。もちろん皆さんがレポートなどを作る時に
これらを真似して「略」とかにしてはダメですよ。

\item 初出の重要語には\underline{下線}をつけました。これらは索引に載っています。

\item 学習の上で勘違いしやすい重要な考え方は, \textgt{太字}で書きました。

\item 特に, 資源生が間違いやすいことを, 「\textgt{よくある間違い}」
として示しました。この部分で間違えると, テキストをきちんと読んで
いないとみなされ, \textgt{成績評価で痛い目に会います}!

\item 各章末に, 「演習問題」を設けました。これらはそれまでの内容を総合的に
使う問題であり, 数学以外の様々な分野への応用例等も盛り込んでいます。
それほど難しくはありませんので, 自力で考え, 楽しみながら取り組んで下さい。
なお. これらの解答は, 原則的に省略しました。高校数学だけを手っ取り早く
自習したいという人は, これらを飛ばしても構いません。

\item 脚注(各ページの下の欄外コメント)は, 理解の補助と, 
大学数学へ橋渡しのためです。もし脚注が理解できなくても, 
本文が理解できればOK。

\item 「証明終わり」を$\blacksquare$という記号で
表します。その他の記号は, 第\ref{chapt_logic}章を参照してください。
\end{itemize}

誤植や間違いを見つけたら, 以下にご連絡下さい:\\
nasahara.kenlo.gw@u.tsukuba.ac.jp\\
訂正は, 次のウェブサイトに掲示します:\\
http://www.agbi.tsukuba.ac.jp/\~{}shigen\_\,remedial/

\textgt{謝辞}

{\small
微分の定義や定式化は, 数理物質科学研究科の西村泰一先生の講義を参考にしました。
「よくある質問」は, 主に平成20年度以降の「基礎数学(I, II)」「数理科学演習」
でのアンケート等から作りました。意見を寄せてくれたり誤植を教えてくれた
受講生とTAの皆さん, 特に山崎一磨君と片木仁君に感謝します。
組版や作図は, LaTeX (Version 3.1415926), emath (Version f051107c), 
GNUPLOT (Version 4.6), LibreOffice (Version 4.2.8.2), Ubuntu Linux 14.04LTSで行いました。\\

平成28年3月23日\\   筑波大学生物資源学類補習担当 奈佐原顕郎}
